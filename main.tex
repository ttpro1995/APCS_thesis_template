% Template KLTN cho SV trường ĐHKHTN
% Liên hệ: nqminh@fit.hcmus.edu.vn
% Last update: 30/11/2016

% Chú ý: đọc các phần chú ý đóng khung của file này và chỉnh lại cho phù hợp.
% Trước khi build, xóa hết các file được tạo ra trong quá trình build trước đó (tất cả các file nằm ở thư mục gốc, trừ file main.tex này), và build theo thứ tự: BIB > PDF > PDF.
% Nếu cập nhật tài liệu tham khảo, cũng cần build lại theo cách trên.

% Đã kiểm tra trên MikTeX 2.9 Windows, TexMaker, Winshell

\documentclass[oneside,a4paper,12pt]{extreport}

% Font tiếng Việt
\usepackage[T5]{fontenc}
\usepackage[utf8]{inputenc}
\DeclareTextSymbolDefault{\DH}{T1}

% Tài liệu tham khảo
\usepackage[
	sorting=nty,
	backend=bibtex,
	defernumbers=true]{biblatex}
\usepackage[unicode]{hyperref} % Bookmark tiếng Việt
\addbibresource{References/references.bib}

\makeatletter
\def\blx@maxline{77}
\makeatother

% Chèn hình, các hình trong luận văn được để trong thư mục Images/
\usepackage{graphicx}
\graphicspath{ {Images/} }

% Chèn và định dạng mã nguồn
\usepackage{listings}
\usepackage{color}
\definecolor{codegreen}{rgb}{0,0.6,0}
\definecolor{codegray}{rgb}{0.5,0.5,0.5}
\definecolor{codepurple}{rgb}{0.58,0,0.82}
\definecolor{backcolour}{rgb}{0.95,0.95,0.92}
\lstdefinestyle{mystyle}{
    backgroundcolor=\color{backcolour},   
    commentstyle=\color{codegreen},
    keywordstyle=\color{magenta},
    numberstyle=\tiny\color{codegray},
    stringstyle=\color{codepurple},
    basicstyle=\footnotesize,
    breakatwhitespace=false,         
    breaklines=true,                 
    captionpos=b,                    
    keepspaces=true,                 
    numbers=left,                    
    numbersep=5pt,                  
    showspaces=false,                
    showstringspaces=false,
    showtabs=false,                  
    tabsize=2
}
\lstset{style=mystyle}

% Chèn và định dạng mã giả
\usepackage{amsmath}
\usepackage{algorithm}
\usepackage[noend]{algpseudocode}
\makeatletter
\def\BState{\State\hskip-\ALG@thistlm}
\makeatother

% Bảng biểu
\usepackage{multirow}
\usepackage{array}
\newcolumntype{L}[1]{>{\raggedright\let\newline\\\arraybackslash\hspace{0pt}}m{#1}}
\newcolumntype{C}[1]{>{\centering\let\newline\\\arraybackslash\hspace{0pt}}m{#1}}
\newcolumntype{R}[1]{>{\raggedleft\let\newline\\\arraybackslash\hspace{0pt}}m{#1}}

% Đổi tên mặc định
%\renewcommand{\chaptername}{Chương}
%\renewcommand{\figurename}{Hình}
%\renewcommand{\tablename}{Bảng}
%\renewcommand{\contentsname}{Mục lục}
%\renewcommand{\listfigurename}{Danh sách hình}
%\renewcommand{\listtablename}{Danh sách bảng}
%\renewcommand{\appendixname}{Phụ lục}

% Dãn dòng 1.5
\usepackage{setspace}
\onehalfspacing

% Thụt vào đầu dòng
\usepackage{indentfirst}

% Canh lề
\usepackage[
  top=35mm,
  bottom=30mm,
  left=35mm,
  right=20mm,
  includefoot]{geometry}
  
% Trang bìa
\usepackage{tikz}
\usetikzlibrary{calc}
\newcommand\HRule{\rule{\textwidth}{1pt}}

% ========================================================================================= %
% CHÚ Ý: Thông tin chung về KLTN - sinh viên điền vào đây để tự động update các trang khác  %
% ========================================================================================= %
\newcommand{\tenSV}{Tên~Sinh~Viên} % Dấu ~ là khoảng trắng không được tách (các chữ nối với nhau bằng dấu ~ sẽ nằm cùng 1 dòng
\newcommand{\mssv}{1234567}
\newcommand{\tenKL}{Sử~dụng~LaTeX trong Khoá~luận~tốt~nghiệp} % Chú ý dấu ~ trong tên khóa luận
\newcommand{\tenGVHD}{Tên~Giáo~Viên}
\newcommand{\tenBM}{Công nghệ tri thức}

\begin{document}

\begin{titlepage}

\begin{center}
UNIVERSITY OF SCIENCE\\
ADVANCED PROGRAM IN COMPUTER SCIENCE\\[2cm]
%BỘ MÔN \MakeUppercase{\tenBM}\\[2cm]

{ \Large \bfseries \MakeUppercase{\tenSV ~-~ \mssv} \\[2cm] } %Nếu có 2 sinh viên thì chèn thêm 1 dòng vào đây, đồng thời chú ý canh trang cho phù hợp

{ \LARGE \bfseries \MakeUppercase{\tenKL} \\[2cm] } %SV có thể tự điền tên vào nếu cần xuống dòng chỗ phù hợp

\Large BACHELOR OF SCIENCE IN COMPUTER SCIENCE\\[2cm]

%\Large GIÁO VIÊN HƯỚNG DẪN\\
%\MakeUppercase{\tenGVHD} \\[2cm]

\begin{tikzpicture}[remember picture, overlay]
  \draw[line width = 2pt] ($(current page.north west) + (2cm,-2cm)$) rectangle ($(current page.south east) + (-1.5cm,2cm)$);
\end{tikzpicture}

\vfill
HO CHI MINH CITY, \the\year

\end{center}

\end{titlepage}

\input{Title/title2.tex}
% Sasu trang Title, các bạn chèn nhận xét gủa GVHD và GVPB. Nhận xét sẽ được giáo vụ phát sau buổi bảo vệ để các bạn đóng quyển.

\pagenumbering{roman} % Đánh số i, ii, iii, ...

%\addcontentsline{toc}{chapter}{Lời cam đoan}
%\include{Appendix/reassurances}

\addcontentsline{toc}{chapter}{ACKNOWLEDGMENTS}
\chapter*{ACKNOWLEDGMENTS}
\label{thanks}

Tôi xin chân thành cảm ơn~\ldots

Xin cám ơn!

\vspace{3cm}
\hspace{7cm}
\begin{minipage}[ht]{0.48\textwidth}
\begin{center}
Tp. Hồ Chí Minh, ngày ... tháng ... năm 2016

Sinh viên thực hiện

\tenSV
\end{center}
\end{minipage}

%\addcontentsline{toc}{chapter}{Đề cương chi tiết}
%\begin{flushleft}
Khoa Công Nghệ Thông Tin\\
Bộ môn \tenBM\\[2cm]
\end{flushleft}

\begin{center}
\LARGE \textbf{ĐỀ CƯƠNG CHI TIẾT}
\end{center}

\begin{tabular}{|L{15cm}|}
\hline
\textbf{Tên đề tài:} \tenKL \\
\hline
\textbf{Giáo viên hướng dẫn:} \tenGVHD \\
\hline
\textbf{Thời gian thực hiện:} 01/01/2000-01/01/2001\\
\hline
\textbf{Sinh viên thực hiện:} \tenSV ~-~ \mssv \\
\hline
\textbf{Loại đề tài:} Tìm hiểu công nghệ (có hoặc không ứng dụng minh hoạ), Xây dựng ứng dụng, ...\\
\hline
\end{tabular}\\[1cm]

\begin{tabular}{|L{15cm}|}
\hline
\textbf{Nội dung đề tài:} mô tả chi tiết nội dung đề tài, yêu cầu, phương pháp thực hiện, kết quả đạt được, ...\\
\hline
\textbf{Kế hoạch thực hiện:} mô tả chi tiết thời gian của các giai đoạn thực hiện và phân công công việc của từng thành viên trong nhóm\\
\hline
\end{tabular}\\[1cm]

\begin{tabular}{C{8cm}C{8cm}}
Xác nhận của GVHD & Ngày ... tháng ... năm 2016 \\
    ~\\~          & ~\\~                        \\
\tenGVHD          & \tenSV                          
\end{tabular}

% Mục lục, danh sách hình, danh sách bảng
\addcontentsline{toc}{chapter}{TABLE OF CONTENTS}
\tableofcontents
\listoffigures
\listoftables

\addcontentsline{toc}{chapter}{ABSTRACT}
\chapter*{ABSTRACT}
\label{tomtat}

Tóm tắt khóa luận: trình bày tóm tắt vấn đề nghiên cứu, các hướng tiếp cận, cách giải quyết vấn đề và một số kết quả đạt được.
Bản tóm tắt dài từ 1 đến 2 trang.

\clearpage

\pagenumbering{arabic} % Đánh số 1, 2, 3, ...

% Các chương nội dung
\chapter{Giới thiệu}
\label{Chapter1}

Template này được xây dựng nhằm giúp sinh viên thuận lợi hơn trong quá trình viết khóa luận tốt nghiệp.
Sử dụng template này, sinh viên chỉ cần tập trung trình bày phần nội dung, không phải quan tâm đến phần định dạng.


\include{Chapter2/chapter2}

\include{Chapter3/chapter3}

% Công trình của tác giả (nếu không có thì comment 02 dòng dưới)
\addcontentsline{toc}{chapter}{Danh mục công trình của tác giả}
\include{Appendix/publish}

% In tài liệu tham khảo
\addcontentsline{toc}{chapter}{TABLE OF CONTENTS}
\printbibheading[title={TABLE OF CONTENTS}]

\printbibliography[heading=subbibliography, title={Vietnamese}, keyword=Viet, resetnumbers=true]

\DeclareNameAlias{sortname}{last-first}
\DeclareNameAlias{default}{last-first}

\printbibliography[heading=subbibliography, title={English}, notkeyword=Viet, resetnumbers=4] 
% ===================================================================== %
% CHÚ Ý: phải gán lại resetnumbers=số tài liệu tham khảo tiếng Việt + 1 %
% ===================================================================== %

% Phần phụ lục
\appendix

\chapter{APPENDICES 1}
\label{Appendix1}

Đây là phụ lục.
\chapter{APPENDICES 2}
\label{Appendix2}

Đây là phụ lục 2.

\end{document} 